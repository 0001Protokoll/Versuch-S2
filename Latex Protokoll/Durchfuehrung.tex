%\documentclass[12pt, letterpaper]{article}

 
%\begin{document}
\section{Durchführung}
Die aus Versuch S1 gewonnenen Daten, der Berechnungen der Energien von Chlorwasserstoff und Kohlenmonoxid in Abhängigkeit von dem Abstand der Kerne wurden nun mittels Igor Pro ausgewertet. Sie entstanden aus der Lösung der zeitunabhängigen Schrödingergleichung mithilfe des Programms -Gaussian- unter Verwendung der Born-Oppenheimer-Näherung mit den Methoden B3LYP und CCSD. 
Hierfür wurden die Daten eingelesen, in die Einheiten Hartee und Centimeter umgerechnet, da diese in der Spektroskopie üblich sind. Die Energie wurde gegen den Abstand der Kerne aufgetragen und nahe dem Minimum mit der Morsefunktion durch Igor Pro intrapoliert. Durch die Vorgabe eines Sinnvollen Bereiches in dem die Werte liegen wurde es dem Programm erleichtert die Parameter an die Daten anzupassen.
Hierbei wurden die Rotationskonstanten und die Anharmonizität unter Verwendung der Massen und der Gleichgewichtsabstände erhalten, woraus die Rotationsschwingungsspektren bei unterschiedlichen Temperaturen (0.01 K, 1 K, 10 K, 100 K, 1000 K) simuliert werden konnten. 
%\end{document}