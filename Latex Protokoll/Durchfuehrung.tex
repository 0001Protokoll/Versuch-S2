%\documentclass[12pt, letterpaper]{article}

 
%\begin{document}
\section{Durchführung}
Die aus Versuch S1 gewonnenen Daten, der Berechnungen der Energien von Chlorwasserstoff und Kohlenmonoxid in Abhängigkeit von dem Abstand der Kerne wurden nun mittels Igor Ausgewertet. Hierfür wurden die Daten eingelesen, in die Einheiten Hartee und Meter umgerechnet und anschließend nahe dem Minimum mit der Morsefunktion intrapoliert. Hierbei wurden die Rotationskonstanten und die Anharminizität unter Verwendung der Massen und der Gleichgewichtsabstände erhalten. Hieraus wurde ein Rotationsschwingungsspektrum simuliert.
%\end{document}