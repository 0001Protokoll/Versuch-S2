%\documentclass[12pt, letterpaper]{article}

 
%\begin{document}
\section{Zusammenfassung}
Durch die Anpassung der in Versuch S1 erhaltenen Werte im Minimum mit der Morsefunktion wurden für CO und HCl die harmonische Konstante, die Anharmonizitätskonstante, der Gleichgewichtsabstand, die Dissoziationsenergie und die Rotationskonstante wie in Tabelle 4.1 dargestellt berechnet. Die erhaltenen Werte stimmen in guter Näherung mit den Literaturwerten überein. Sämtliche Werte von CO liegen wie erwartet unter den jeweiligen Werten HCl. Dies spiegelt sich auch in den simulierten Spektren wider. So liegt die Lücke von CO bei 1000 Kelvin bei rund 1940 inversen Centimetern, wohingegen sie bei HCl bei über 2500 liegt. Je höher die Temperatur in der Simulation gewählt wurde, desto höher mehr Linien und somit mehr Übergänge sind erkennbar.
%\end{document}