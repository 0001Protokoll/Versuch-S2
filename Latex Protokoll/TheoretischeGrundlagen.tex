%%\documentclass[a4paper, 12pt]{scrreprt}

\documentclass[a4paper, 12pt]{scrartcl}
%usepackage[german]{babel}

%\usepackage{amsmath}
%usepackage{color}
\usepackage[utf8]{inputenc}
\usepackage[T1]{fontenc}
\usepackage{wrapfig}
\usepackage{lipsum}% Dummy-Text

%%%%%%%%%%%%bis hierhin alle nötigen userpackage

\usepackage[utf8]{inputenc}
\usepackage{amsmath}
\usepackage{amsfonts}
\usepackage{amssymb}
\usepackage{graphicx}
%\usepackage{wrapfig}
\usepackage[ngerman]{babel}
\usepackage[left=25mm,top=25mm,right=25mm,bottom=25mm]{geometry}
%\usepackage{floatrow}
\setlength{\parindent}{0em}
\usepackage[font=footnotesize,labelfont=bf]{caption}
\numberwithin{figure}{section}
\numberwithin{table}{section}
\usepackage{subcaption}
\usepackage{float}
\usepackage{url}
\usepackage{fancyhdr}
\usepackage{array}
\usepackage{geometry}
%\usepackage[nottoc,numbib]{tocbibind}
\usepackage[pdfpagelabels=true]{hyperref}
\usepackage[font=footnotesize,labelfont=bf]{caption}
\usepackage[T1]{fontenc}
\usepackage {palatino}
%\usepackage[numbers,super]{natbib}
%\usepackage{textcomp}
\usepackage[version=4]{mhchem}
\usepackage{subcaption}
\captionsetup{format=plain}
\usepackage[nomessages]{fp}
\usepackage{siunitx}
\sisetup{exponent-product = \cdot, output-product = \cdot}
			
%\begin{document}

\section{Theoretische Grundlagen}
Formalisierte Spektroskopie kann die Absorption unseres quantenmechanischen Systems, in unserem Beispiel die Moleküle Kohlenstoffmonoxid und Chlorwasserstoff, als eine Interatkion der Wellenfunktion dieser Systeme mit dem Photon (ebenfalls als eine Funktion darstellbar) beschreiben. Durch diese Überlegung folgt die für die Chemie relevante Bedinung für die Existenz von Rotationsspektren -- nur Moleküle mit permanenten Dipolmoment besitzen ein Rotationsspektrum.\\
\\
Allgemein gilt ferner, dass für einen photoneninduzierten Übergang ausgehend von einem Photon der Energie $E = h \nu$ die Resonanzbedingung : \\
\begin{equation}
E_n - E_m \approx h \nu
\end{equation}
gelten muss. Also die eingestrahlte Energie der Energiedifferenz der betreffenen diskreten Zustände entspricht, um den Übergang vom Zustand $n$ in den Zustand $m$ zu ermöglichen.\\
\\
Betrachtet wir nun die Rotationsspektroskopie, ergibt sich durch das Modell des starren Rotators eine Diskretion dieser Energiewerte in der Einheit $[cm^{-1}]$ gemäß :
\begin{equation}
\frac{E(J)}{hc} =F(J) = BJ(J+1)\quad \quad\quad\quad\quad\textit{wobei\,:}\quad B=\frac{h}{8\pi^2cI}
\end{equation}
Die Auswahlregel für Rotationsspektren lautet $\Delta J = \pm 1$ und ergibt sich aus der bereits vorgestellten quantenmechanischen Beschreibung. Umformen des Energietherms $F(J)$ unter Berücksichtung der Auswahlregel führt zu der allgemeinen Wellenzahl eines Rotationsüberganges als: 
\begin{equation}
\tilde{\nu} = 2B(J+1)
\end{equation}
Mit diesen Informationen haben wir eine Möglichkeit einen direkte Zusammenhang von makroskopischen Größen wie experimentellen Messungen, zu einer mikroskopischen Größe in Form des Trägheitsmoments, ferner den Bindungsabstand bei bekannten Atommassen, zu erhalten.\\
\\
\\
Für die Schwingunsspektropskopie kann in Analogie zum Vorgehen bei der Rotationsspektroskopie durch das Modell des harmonischen bzw. anharmonischen Osszillators Energietherme, Auswahlregeln und Korrelare zwischen makroskopischen Messgrößen und mikroskopischer Eigenschaft des betreffenden Systems, gefunden werden.\\
So ergibt sich im Fall des anharmonischen Osszillators, der die Realität einer chemischen Bindung gut approximiert, Energietherme als :
\begin{equation}
\frac{E(v)}{h} = G(v) = \tilde{\nu}_e(v+\frac{1}{2})-\tilde{\nu}_ex_e(v+\frac{1}{2})^2
\end{equation}
wobei $x_e$ als Koeffizient des quadratischen Therms die Anharmonizitätskonstante beschreibt. Die Auswahlregeln lauten $\Delta v = \pm 1, \pm 2, \pm 3, ... $. Die Differenz zweier Energiewerte ist durch den quadratischen Therm nicht mehr konstant, sondern : 
\begin{equation}
\Delta G = G(v')-G(v'') = \tilde{\nu}_e[1-2x_e(v+1)]
\end{equation}
Wird nun die Differenz nach der Quantenzahl $v$ abgeleitet, folgt eine Konstante als Differenz der Differenzen gemäß : 
\begin{equation}
\Delta\Delta G = -\tilde{\nu}_ex_e
\label{DeltaG}
\end{equation}
Insbesondere Gleichung \ref{DeltaG} ermöglicht uns den zuvor angesprochenen Zusammenhang zwischen makroskopischer und mikroskopischer Welt. Durch eine weitere Überlegung bezüglich der Dissoziation können somit wichtige, zum Teil nicht messbare, Parameter eines Moleküls bestimmt werden. Über geschickte Extrapolation kann die messbare Dissoziationsenergie $D_0$ unter berücksichtung von $v_m$ als maximale Anzahl von Schwingungszuständen vor der Dissoziation durch : 
\begin{equation}
D_0 = \sum_{0}^{v_m-1}\tilde{\nu}_e[1-2x_e(v+1)]=\frac{\tilde{\nu}_e}{4x_e}-\frac{1}{2}\tilde{\nu}_e
\end{equation}
berechnet werden. Ferner ergibt sich die nicht messbare größe der Dissoziation aus dem absoluten Minimum der Morsefunktion, also im Gleichgewichtsabstand $D_e$ als : \\
\begin{equation}
D_e = \frac{\tilde{\nu}_e}{4x_e}-\frac{1}{4}\tilde{\nu}_ex_e\approx\frac{\tilde{\nu}_e}{4x_e}
\end{equation}
\\
Betrachten wir nun die Energie eines zweiatomigen Moleküls in Abhängigkeit des Bindungsabstandes, so kann durch den hinreichend bekannten funktionaler Ausdruck des Morsepotentials : \\
\begin{equation}
V(r) = D_e[1-e^{-\beta(r-r_e)}]^2 \quad\quad\quad \textit{wobei\,:}\quad \beta = \sqrt{\frac{2\pi^2c\mu}{D_eh}}
\end{equation}
die Energie im Molekül angenähert beschrieben werden. Die uns experimentell zugänglichen Größen helfen uns eine solche Funktion idealer zu modellieren, und ferner die Eigenschaften des Systems tiefgreifender zu verstehen. Ebenfalls lassen sich somit Theorien entwickeln und evalidieren, um experimentelle Größen von nicht Modellsystemen vorherzusagen, was eine Einschätzbarkeit von Experimenten impliziert.
%\end{document}