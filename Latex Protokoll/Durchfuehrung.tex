%\documentclass[12pt, letterpaper]{article}

 
%\begin{document}
\section{Durchführung}
Die aus Versuch S1 gewonnenen Daten, der Berechnungen der Energien von Chlorwasserstoff und Kohlenmonoxid in Abhängigkeit von dem Abstand der Kerne wurden nun mittels Igor Pro ausgewertet. Sie entstanden aus der Lösung der zeitunabhängigen Schrödingergleichung mithilfe des Programms -Gaussian- unter Verwendung der Born-Oppenheimer-Näherung mit den Methoden B3LYP und CCSD. 
Hierfür wurden die Daten eingelesen und von der Einheit Hartee in Wellenzahl umgerechnet. Die Energie wurde gegen den Abstand der Kerne aufgetragen und nahe dem Minimum mit der Morsefunktion durch Igor Pro intrapoliert. Hierbei wurden sowohl der Gleichgewichtsabstand als auch die Dissoziationsenergie durch das absolutes Minimum erhalten und somit die Rotationskonstanten und die Anharmonizität unter Verwendung der bekannten Massen errechnet werden. Mit diesen relevanten Größen konnte mithilfe des bereitgestellten Programmpakets \textit{RotVib} Rotationsschwingungsspektren bei unterschiedlichen Temperaturen (0.01 K, 1 K, 10 K, 100 K, 1000 K) simuliert werden. 
%\end{document}